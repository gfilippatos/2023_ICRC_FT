% Please make sure you insert your
% data according to the instructions in PoSauthmanual.pdf
\documentclass[a4paper,11pt]{article}
\usepackage{pos}

\title{EUSO-SPB2 Fluorescence Telescope in-flight performance and preliminary results }
 \ShortTitle{EUSO-SPB2 FT}
 \author*[a]{G. Filippatos}
 \affiliation[a]{Colorado School of Mines, Golden, USA}
 \emailAdd{gfilippatos@mines.edu}
 \onbehalf{for the JEM-EUSO collaboration} 

\abstract{
	The Extreme Universe Space Observatory on a Super Pressure Balloon II (EUSO-SPB2) is scheduled for launch from Wanaka, New Zealand in April/May 2023. 
	Consisting of two optical telescopes, EUSO-SPB2 will search for very high energy neutrinos (E>PeV) via Cherenkov radiation, and ultra high energy cosmic rays (UHECRs, E>EeV) via ultraviolet fluorescence. 
	Building on the EUSO-Balloon (2014) and EUSO-SPB1 (2017) missions, the Fluorescence Telescope (FT) comprises 108 multi-anode photomultiplier tubes at the focus of a one meter entrance diameter Schmidt telescope. 
	The FT will point down at the atmosphere below the SPB’s altitude of 33 km. The mission duration could reach up to 100 days. 
	Prior to flight, the instrument was extensively tested in the laboratory and in the field. 
	These measurements, combined with extensive simulations lead to an expected peak energy sensitivity around 3 EeV. 
	Combined with a three-times-larger field-of-view than previous EUSO balloon missions, this results in an expected observation rate of one UHECR shower per ten hours of observation. 
	The FT is expected to perform the first measurement of UHECRs via fluorescence from sub-orbital space. 
	These observations will serve as a stepping stone to future satellite-based missions, such as Probe of Extreme Multi-Messenger Astrophysics, with enormous exposure to the highest energy UHECRs with all sky coverage.
	In this contribution we will discuss the performance of the FT in flight as well as preliminary results. 
}

\ConferenceLogo{PoS_ICRC2023_logo.pdf}

\FullConference{
38th International Cosmic Ray Conference (ICRC2023)\\
  26 July - 3 August, 2023\\
  Nagoya, Japan
}

\begin{document}
\maketitle

\section{Instrument Overview}

The Fluorescence Telescope (FT) on board EUSO-SPB2 built on the experience of previous EUSO missions.
EUSO-Balloon \citep{EUSO-Ballon} a one night flight from Timmins Canada in 2015 and EUSO-SPB1 \citep{SPB1}, a planned long duration flight from Wanaka New Zealanad in 2017, each flew one photo-detection modules (PDMs) and refractive optics.
The FT was comprised of three PDMs at the focus of a modified Schmidt telescope. 

Each PDM was made up of 36 multi-anode photo-multiplier tubes (MAPMTs Hamamatsu Photonics R11265-M64) woth 64 pixels each. 
Four MAPMTs are grouped together and potted in a gelatanous compound to form an Elementary Cell (EC) with a shared high voltage power supply (HVPS) utilizing a Cockroft-Walton circuit. 
Each MAPMT is read out by a SPACIROC3 ASIC preforming both photon counting and continous integration (KI). 
The photon counting channel of the 2,304 total pixes are digitized with a 952 kHz frequency and a double pulse resolution of \~6 ns.
While the continous integration channel the MAPMTs are digitized at the same cadence, 8 photon counting pixels are grouped together, resulting in 8 KI channels per MAPMT.
Data from the 36 ASICs are sent to three cross boards containing an Atrix 7 FPGA which multiplex the data.
The PDM is controlled by a Xilinx Zynq 7000 FPGA with an embedded dual core ARM9 CPU on a custom readout board which handles triggering and data packaging among other tasks \citep{SPB2Trigger}. 

A dataprocessor containing redundant CPUs, reduandant differential GPSs, a housekeeping board and a clock board, controls the PDMs. 
The clockboard syncronizes the readout of the three PDMs by recieivng the output of the trigger logic on each Zynq board and issuing signals to the three PDMs in parrallel. 
Additionally, data from two differential GPS are packaged with trigger and deadtime information for each event that is recorded by the clockboard. 
The CPU handles commanding of all subsytems as well as the combining of data from the three PDMs and the clockboard. 
In addition to internal commanding, the CPU recieves and processes commands from ground. 
Further, the CPU compresses data for download and transmits all monitoring data to ground. 
Monitoring data includes 18 temperature probes, which were placed around the payload, humidity and pressure sensors, gyroscope information, and data from two photo-diodes at the focus of the telescope. 
There are two calibrated health-LEDs beneath the PDMs which are used to monitor the status of the instrument. 

The three PDMs are at the focus of a 1-m diameter entrance pupil Schmidt telescope, behind field flatteners and BG-3 filters. 
Comprised of six segmented spherical mirrors and an aspheric corrector plate, the telescope focuses light into a spot size of \~2 mm with an optical throughput of 67\%, far greater than refractive optics flown on previous EUSO missions. 
The FT is hung on one half of the EUSO-SPB2 gondola, beneath the science information package (SIP).
Three indpendent telemetry links are handeld by the SIP, with a fourth, a Starlink connection, connected directly to the FT CPU.  
Next to the FT hangs the Cherenkov telescope, with a tilt mechanism allowing for +5/-13 degrees of pointing, beneath the gondola control computer and science power system. 
The 5,025 lbs of instrumentation and support equipment, with an additional 600 lbs of ballasts, were suspended from an 18.8 million cubic foot super pressure balloon.


\section {Pre-Flight Characterizations}

Before the instrument was fully assembled, the PDMs were extensively characterized in the lab using calibrated light sources. 
The telescope was characterized using a 1-m parralell beam system and photodiodes. 

Prior to flight, in August of 2022, the instrument was field tested in Delta Utah, outside of the Telesceop Array Black Rock Mesa FD. 
The signal from extensive air showers (EAS) was mimicked by a frequeny trippled Nd-YAG laser at 355 nm, with a robust automated pointing system. 
Calibrated LEDs placed both in the near and far fields were used to provide an alternative calibration to the piecewise method used in the lab.
Night sky backroung was observed for tens of hours, including stars, meteors, cosmic rays and airglow. 

Immediately before flight, the telescope was tested again in Wanaka New Zealand. 
The same calibrated LEDs used in Delta were utilized and the night sky was observed.

\section{Flight Summary}

\section{Detector Preformance}

\section{Exposure and Acceptance}

\section{Analysis of Events}

\section{Conclusions}
I wish we had SuperBIT's balloon. 

\bibliographystyle{plain}
\bibliography{references}
% \begin{thebibliography}{99}


% \end{thebibliography}



\end{document}
